En esta primera clase aprenderas la sintaxis de html tomand en cuenta que html
 esta compuesto por etiquetas
 de apertura y de cierre y atributos que lo acompañan 


 Que son los atributos

 En HTML, un "atributo" es una característica adicional que se agrega
  a una etiqueta para proporcionar información adicional sobre el elemento.
   Los atributos pueden modificar el comportamiento o la apariencia del elemento 
   al que se aplican. Por ejemplo, en la etiqueta de imagen `<img>`, el atributo
    `src` se utiliza para especificar la ruta de la imagen que se mostrará, 
    mientras que el atributo `alt` se utiliza para proporcionar un texto 
    alternativo en caso de que la imagen no se pueda cargar. Los atributos 
    en HTML se definen dentro de la etiqueta de apertura del elemento y se
     componen de un nombre y un valor, separados por un signo igual, como 
     en el siguiente ejemplo: `<etiqueta atributo="valor">`.

     <etiqueta></etiqueta>
     <apertura></cierre>
     <inicio atributo ="valor"></cierre>

     <inicio atributo ="valor">contenido</cierre>

     comentarios 

     <!-- este es un comentario comentarios-->

     ¿Qué es HTML?
HTML significa lenguaje de marcado de hipertexto
HTML es el lenguaje de marcado estándar para crear páginas web.
HTML describe la estructura de una página web.
HTML consta de una serie de elementos.
Los elementos HTML le dicen al navegador cómo mostrar el contenido.
Los elementos HTML etiquetan fragmentos de contenido como "este es un encabezado", "este es un párrafo", "este es un enlace", etc.




Ejemplo explicado
La <!DOCTYPE html>declaración define que este documento es un documento HTML5
El <html>elemento es el elemento raíz de una página HTML.
El <head>elemento contiene metainformación sobre la página HTML.
El <title>elemento especifica un título para la página HTML (que se muestra en la barra de título del navegador o en la pestaña de la página)
El <body>elemento define el cuerpo del documento y es un contenedor de todos los contenidos visibles, como encabezados, párrafos, imágenes, hipervínculos, tablas, listas, etc.
El <h1>elemento define un encabezado grande.
El <p>elemento define un párrafo.

¿Qué es un elemento HTML?
Un elemento HTML se define mediante una etiqueta de inicio, algo de contenido y una etiqueta de cierre:

< nombre de etiqueta > El contenido va aquí... < /nombre de etiqueta >
El elemento HTML es todo, desde la etiqueta inicial hasta la etiqueta final:

< h1 > Mi primer título < /h1 >
< p > Mi primer párrafo. < /p >

Encabezados HTML
Los encabezados HTML se definen con las etiquetas <h1>to <h6>.

<h1>Define el título más importante. <h6>define el encabezado menos importante: 
     


